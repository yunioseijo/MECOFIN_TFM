\section{Estado del Arte}


\subsection{Aprendizaje automático}
Desde la aparición de las computadoras estas han sido capaces de resolver problemas muy complejos para el hombre, pero aún no poseen la habilidad de aprender por sí solas. A pesar de esto, el desarrollo de la inteligencia artificial ha propuesto un gran número de algoritmos que intentan imitar esta habilidad, los cuales han demostrado ser especialmente efectivos para ciertos tipos de problemas.

El aprendizaje automático o aprendizaje de máquina (del inglés Machine Learning), es un campo multidisciplinario cuyo objetivo es desarrollar programas de computadora que mejoren su funcionamiento en ciertas tareas a partir de la experiencia \cite{Mitchell1997}.
La minería de datos ha contribuido al desarrollo del aprendizaje automático ya que sus métodos han sido ampliamente utilizados en el descubrimiento de información valiosa a partir de datos almacenados\cite{Witten2002}. Con frecuencia el campo de aplicación del aprendizaje automático se solapa con el de la estadística, ya que las dos disciplinas se basan en el análisis de datos, por lo que resulta difícil. No obstante, el aprendizaje automático se centra más en el estudio de la complejidad computacional de los problemas. Muchos problemas son de clase NP-Hard, por lo que gran parte de la investigación realizada en esta rama de la ciencia se enfoca al diseño de soluciones factibles a esos problemas.

De forma más concreta, se trata de crear algoritmos capaces de generalizar comportamientos a partir de información suministrada en forma de ejemplos. Tales ejemplos sirven como entrenamiento, para que luego el algoritmo pueda enfrentarse a nuevos datos. Estos algoritmos construyen un modelo a partir de los ejemplos y lo usan para hacer predicciones, en lugar de seguir instrucciones estáticas estrictas como cualquier otro algoritmo.

Existen varias formas de adquirir el conocimiento necesario. Una puede ser directamente a partir del humano, o a partir de problemas resueltos previamente. Los datos que se le proporcionan al programa permiten que el algoritmo de aprendizaje sea capaz de extraer de ellos la información necesaria para enfrentarse a nuevos datos y realizar la función para la cual fue diseñado. Podemos definir un ejemplo de entrenamiento de la siguiente forma:

Definición 1.1. Se denomina instancia o caso X a la representación de un determinado concepto u objeto. Las instancias se suelen representar como un conjunto de atributos (x1; : : : ; xn) 2 X.
Los atributos pueden tomar tanto valores numéricos como nominales. Esta representación de instancia es una abstracción de los objetos, pudiéndose ignorar otras características que no son representadas por los atributos.

Según el resultado que se desea obtener a partir de un sistema, existen varias categorías en las cuales se engloban las tareas de aprendizaje automático. A continuación, se describen algunas de las más importantes:

- Clasificación: la entrada es dividida en dos o más clases, y el sistema debe producir un modelo capaz de asignarle a una nueva entrada una o más de estas clases. Típicamente se lleva a cabo mediante aprendizaje supervisado.

- Regresión: es también una tarea supervisada, similar a la anterior pero la salida obtenida es continua en lugar de discreta.

- Agrupamiento: el conjunto de entrada es dividido en grupos. A diferencia de la clasificación los grupos no son conocidos de antemano, haciendo de esta una tarea no supervisada.

- Descubrimiento de reglas: se trata de encontrar reglas, generalmente del tipo if -then, que relacionen a los datos.

El aprendizaje automático tiene innumerables aplicaciones varias ramas de la ciencia. Por ejemplo, en bioinformática para el diagnóstico médico o la clasificación de secuencias de ADN, en economía para el análisis del mercado de valores o la detección de fraudes en tarjetas de crédito, o en el reconocimiento del habla y del lenguaje escrito en campos como la teoría de juegos, robótica, etc.\cite{Mitchell1997}

 


\subsection{Redes neuronales Artificiales}
Las Redes Neuronales Artificiales o ANNs se agrupan dentro de las técnicas conexionistas de la inteligencia artificial y constituye una de las áreas de estudio más ampliamente difundidas. Constituyen modelos matemáticos inspirados en la biología que han sido utilizados para dar solución a problemas complejos, pues son considerados excelentes aproximadores de funciones no lineales. Las ANNs son capaces de aprender las características relevantes de un conjunto de datos para luego reproducirlas en entornos ruidosos o incompletos, siendo especialmente útiles para tareas de clasificación y regresión.\cite{Hammer2003}

\subsection{Redes neuronales recurrentes}
\lipsum[2]

\subsection{Métricas de rendimiento}
\lipsum[2]
\clearpage

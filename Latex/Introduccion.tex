\section{Introducción}
 
\subsection{Motivación}
Comencé a estudiar el master con la idea de aprender sobre técnicas para predecir el valor de una acción de mercado, asignaturas como modelos predictivos, optimización, minería de datos, técnicas de aprendizaje automático resonaban en mi cabeza como una gran oportunidad para aprender y desarrollar una idea que había comenzado Cuba, donde programé algunas estrategias de compra y venta basado en el análisis técnico, utilicé lo que tenía al alcance y era más común en el 2020 con el boom de la expansión de las criptomonedas, por todos lados aparecían historias de cómo hacer dinero usando análisis técnico (medias móviles, bandas de bollinger, RSI, puntos pivotes, análisis de velas japonesas y un poco de chartismo donde se dibujaban tendencias con líneas, se pintan triángulos y banderas intentando predecir qué pasará mañana. El sueño de lograr libertad financiera, lo que para mí significa tener dinero suficiente para trabajar por placer en esas ideas que surgen para cambiar la vida y no por dinero, disponer de tiempo para compartir con la familia y amigos, poder ayudar a personas necesitadas con donaciones, algo que aprendí leyendo sobre las personas más ricas del mundo, donan millones a causas humanitarias y la filantropía es parte de sus paciones. Como el caso del gran inversor Warren Buffett para muchos considerado el mejor inversor de todos los tiempos quien ha donado 41 millones de dólares a causas benéficas. También la historia de Jim Simmons nombrado el hombre que descifró el mercado, un matemático que comenzó su carrera en las inversiones financieras comenzando como solo un pasatiempo para él, hasta que decidió dedicarse 100% a este trabajo y reunió en su fondo de inversión a muchos especialistas que no tenían origen financiero, incluyendo a matemáticos, físicos, y expertos en estadística, en el caso de Simmons dona millones al año para pagar a profesores de matemáticas en escuelas públicas. Todas esa son las causas que me impulsan a plantearme un nuevo reto e intentar predecir el precio de algunas acciones de las empresas mejor valoradas de los Estados Unidos de América según su capitalización de mercado utilizando como datos el valor de cierre de las acciones del activo en cuestión que estoy analizando y otras variables como el precio de otros activos del mismo sector, el petróleo por su importancia y dependencia energética de la economía actual y el índice en al que pertenece. Escojo múltiples empresas para tener un portafolio diversificado por sectores estratégicos de la economía, Como los sectores de las finanzas, las tecnologías, la energía, salud, construcción, entre otras porque son en mi criterio las que han revolucionado el mundo y hacen que una mayor productividad, en general, produzca una expansión económica y mayor bienestar en la sociedad, para el análisis y predicción usaré una técnica de inteligencia artificial, redes neuronales recurrentes estudiada en una de las asignaturas que más interés causó en mí, Redes Neuronales y Deep Learning.
\subsection{Objetivos generales y específicos}
Determinar si el valor de cierre de una acción subirá o bajará al día siguiente utilizando diferentes técnicas de aprendizaje automático y redes neuronales artificiales.
\clearpage
